\documentclass[wide, 11pt]{mwart}

\usepackage[utf8]{inputenc}
\usepackage[OT4,plmath]{polski}
\usepackage{amsmath,amssymb,amsfonts,amsthm,mathtools}
\usepackage{algorithmicx} 
\usepackage{algpseudocode}
\usepackage{graphicx}

\date{Wrocław, \today}
\title{Pracownia z analizy numerycznej (M),\\[8pt]\large{Zadanie P3.11}}
\author{Marcin Grzywaczewski\footnote{killavus@gmail.com} \and Szymon Koper\footnote{sz.koper@gmail.com}}

\begin{document}
	\maketitle
	\begin{center}
	Numery indeksów:\\247948 (MG)\\247985 (SzK)
	\end{center}
	\thispagestyle{empty}
	\newpage

\section{Wstęp}
Wielomiany są niezwykle ważną z numerycznego punktu widzenia rodziną funkcji.
Ich unikalne własności pozwalają na wykorzystanie ich w wielu dziedzinach -
od aproksymacji i interpolacji funkcji, po obliczanie wartości całek za pomocą
metod numerycznych. Posiadają one wiele postaci, każda z nich charakteryzuje
się innymi wadami, oraz zaletami.

W tej pracowni zostaną omówione dwie postacie wielomianów - najbardziej 
popularna potęgowa, oraz postać Béziera. Skupiono się na sprawdzenie pod kątem
numerycznym sposobów wzajemnych przekształceń tych postaci z jednej w drugą,
oraz problem obliczania wartości wielomianu w punkcie $x$ w obu tych 
postaciach.

\subsection{Postać potęgowa wielomianu}
Znana jest bardzo prosta metoda obliczania wartości wielomianu w postaci
potęgowej. Efektywny czasowo algorytm schematu Hornera pozwala na obliczanie
wartości w punkcie $x$ wielomianu za pomocą $n$ mnożeń i $n$ dodawań.

Algorytm opiera się na prostej obserwacji. Wielomian $\pi$ $n$-tego stopnia
\[\pi_n(x) = a_nx^n + a_{n-1}x^{n-1} + \ldots + a_1x + a_0\] możemy zapisać 
jako:
\begin{equation}
  \label{horner_scheme_formula}
  \pi_n(x) = a_0 + x(a_1 + x(a_2 + x(\ldots + x(a_{n-1} + a_nx))\ldots)
\end{equation}

Oczywiście, dla wielomianu $0$ stopnia jego wartością dla dowolnego punktu $x$
jest $a_0$. W przypadku wielomianów stopnia wyższego, mając dane współczynniki
$a_n, a_{n-1}, \ldots, a_1, a_0$, oraz punkt $x$, opierając się na wzorze 
(\ref{horner_scheme_formula}) możemy skonstruować następujący związek 
rekurencyjny:
\begin{subequations}
  \begin{align}
    \beta_0 = \pi_n(x)\\
    \beta_0 = \beta_1 + a_0\\
    \beta_n = a_{n-1} + a_nx\\
    \beta_i = x(a_{n-2} + \beta_{i+1})
  \end{align}
\end{subequations}

Obliczając wartości $\beta_n, \beta_{n-1}, \ldots, \beta_1, \beta_0$ otrzymamy
algorytm obliczania wartości wielomianu w postaci potęgowej w czasie liniowym,
znany pod nazwą \emph{schematu Hornera}.

Niestety, obliczanie wartości wielomianu w postaci potęgowej jest 
słabo uwarunkowane. Sprawia to, że mimo tego, iż posiadamy tak dobry algorytm,
jakim jest schemat Hornera, błąd względny obliczania wartości $\pi_n(x)$ będzie 
duży.

\subsection{Postać Béziera wielomianu}

Wielomian $\pi_n$ możemy także przedstawić w postaci Béziera, tj. przedstawić
go jako kombinację liniową w bazie wielomianów Bernsteina, które wyrażają się
wzorem:
\begin{equation}
  B^n_i(u) = {n \choose i}u^i(1-u)^{n-i}
\end{equation}

Wielomiany te służą także do przedstawiania szeroko stosowanych w grafice
komputerowej krzywych Béziera.

W wypadku takiego przedstawienia wielomian $\pi_n$ wyraża się wzorem:
\begin{equation}
  \label{bezier_form}
  \pi_n(x) = \sum_{i=0}^n\beta_iB^n_i(x)
\end{equation}

Do obliczania wartości wielomianu (\ref{bezier_form}) w punkcie $x$ wykorzystamy
następujący związek rekurencyjny:
\begin{subequations}
  \begin{align}
    \gamma_i^{(0)} := \beta_i\\
    \gamma_i^{(k)} := (1-x)\gamma_i^{(k-1)} + t\gamma_{i+1}^{(k-1)}\\
    \gamma_0^{(n)} = \pi_n(x)
  \end{align}
\end{subequations}
Obliczając wartości w sposób przedstawiony na schemacie:
\begin{subequations}
  \begin{gather}
    \gamma_0^{(0)}, \gamma_1^{(0)}, \gamma_2^{(0)}, \gamma_3^{(0)}, \ldots, \gamma_n^{(0)}\\
    \gamma_0^{(1)}, \gamma_1^{(1)}, \gamma_2^{(1)}, \ldots, \gamma_{n-1}^{(1)}\\
    \vdots \\
    \gamma_0^{(n-1)}, \gamma_1^{(n-1)}\\
    \gamma_0^{(n)}
  \end{gather}
\end{subequations}

Otrzymujemy działający w czasie kwadratowym algorytm obliczania wartości
wielomianu $\pi_n$ w punkcie $x$ w postaci Béziera, zwany 
\emph{algorytmem de Casteljau}.

Obliczanie wartości wielomianu w postaci Béziera jest lepiej uwarunkowane od
obliczania wartości wielomianu w postaci potęgowej. Jest to nasza główna
motywacja stojąca za przyjrzeniem się zagadnieniu wzajemnych przekształceń tych
postaci.

\subsection{Przekształcanie wielomianu z postaci potęgowej do postaci Béziera}
Przekształcanie postaci potęgowej do postaci Béziera realizujemy poprzez użycie
następującego wzoru:
\begin{equation}
  \label{to_poly}
  t^i = {n \choose i}^{-1}{\sum_{j=i}^n}{j \choose i}B^n_j(t) 
\end{equation}
Oraz:
\begin{equation}
  B^n_i(t) = {\sum_{k=i}^n}(-1)^{i+k}{k \choose i}{n \choose k}t^k
\end{equation}
Stosujemy te wzory w następujący sposób. Mając daną postać potęgową wielomianu 
\[\pi_n(t) = {\sum_{i=0}^n}a_it^i\], aby obliczyć wartości $\beta_0, \beta_1, 
\beta_2, \ldots, \beta_n$ postaci Béziera tego samego wielomianu 
\[\pi_n(t) = {\sum_{i=0}^n}\beta_iB^n_i(t)\] możemy podstawić równanie 
(\ref{to_poly}) w miejsce $t^i$:
\begin{equation}
  \pi_n(t) = 
  {\sum_{i=0}^n}a_i{n \choose i}^{-1}{\sum_{j=i}^n}{j \choose i}B^n_j(t)
\end{equation}
\newpage
Korzystając z równości powyżej można skonstruować algorytm na wyznaczanie 
$\beta_0, \beta_1, \beta_2, \ldots, \beta_n$
dla wielomianu $\pi_n$ w następujący sposób:
\begin{algorithmic}
  \For {$i = 0 \to n$}
    \For {$j = i \to n$}
      \State $\beta_j \gets a_i{n \choose i}^{-1}{j \choose i}$
    \EndFor  
  \EndFor
\end{algorithmic}
Co ciekawe, iloczyn \[{n \choose i}^{-1}{j \choose i}\] możemy zapisać jako:
\[{\prod^n_{k=n-i+1}}\frac{k-n+j}{k}\]
Co zostało użyte w implementacji.

\subsection{Przekształcanie wielomianu z postaci Béziera do postaci potęgowej}
Przekształcenie z postaci potęgowej do postaci Béziera realizujemy korzystając
z następującego wzoru:
\begin{equation}
  \label{to_bez}
  B^n_i(t) = {\sum_{k=i}^n}(-1)^{i+k}{k \choose i}{n \choose k}t^k
\end{equation}

Podstawiając równość (\ref{to_bez}) do naszej postaci Béziera wielomianu 
$\pi_n$ otrzymujemy następujący wzór:
\begin{equation}
  \pi_n(t) = {\sum_{i=0}^n}\beta_i{\sum_{k=i}^n}(-1)^{i+k}
  {k \choose i}{n \choose k}t^k
\end{equation}

Co pozwala nam na konstrukcję następującego algorytmu:
\begin{algorithmic}
  \For {$i = 0 \to n$}
    \For {$k = i \to n$}
      \State $a_i \gets (-1)^{i+k}{n \choose i}{k \choose i}\beta_i$
    \EndFor
  \EndFor
\end{algorithmic}

Niestety, nie znaleziono żadnych możliwości uproszczenia iloczynu 
${n \choose i}{k \choose i}$. Musimy go liczyć wprost. W implementacji tego rozwiązania wykorzystano trójkąt Pascala.

\section{Testy}
\subsection{Wielokrotnie przekształcanie z jednej bazy do drugiej}
Zaprezentowane wcześniej metody zostały przetestowane na czterech zaprezentowanych poniżej wielomianach. Wielomiany te zostały otrzymane z wielomianów o stopniach mniejszych o 1 przez pomnożenie ich razy $(x-x_0)$. Zatem $x_0$ jest jednym z pierwiastków. W przypadku pierwszego równania jest to $1, 4, -3, 2$. Podane poniżej wielomiany są $4, 10, 12, 20$ stopnia. Nawet przy tak niskich stopniach konwersja między postaciami jest stratna. 
\begin{equation}
  \begin{align}
    p1(x) = x^{4} - 6x^{3} + 11x^{2} - 6x\\
    p2(x) = 2x^{10}-31x^{9}+44x^{8}+204x^{7}+13x^{6}-287x^{5}+159x^{4}+60x^{3}-29x^{2}-48x+144\\
    p3(x) = -3x^{12}-8x^{11}+7x^{10}+4x^{9}-40x^{8}-60x^{7}-54x^{6}-18x^{5}+109x^{4}-2x^{2}+29x+24\\
    p4(x) = -7x^{20}+62x^{19}-79x^{18}+16x^{17}-97x^{16}-4x^{15}-11x^{14}+28x^{13}+25x^{12}-193x^{11}\\
    +285x^{10}-226x^{9}+13x^{8}-6x^{7}-7x^{6}-15x^{5}-51x^{4}+168x^{3}-26x^{2}+4x-16
  \end{align}
\end{equation}
Przeprowadzono test, przekształcając w postać Béziera wielomiany $p1,p2,p3,p4$ kolejno $1,10,100,1000,10000,100000,1000000$ razy. Ewaluacja algorytmami de Casteljau i Hornera wykazała te same wyniki i była przeprowadzana w punkcie zerowym wielomianu. Program wykonujący ten test został uwzględniony w plikach pracowni.
Poprzez obliczenie wartości w pierwiastkach wielomianów zostały uzyskane błędy bezwzględne. Wykresy to ilustrujące są widoczne poniżej.

\includegraphics*[width=\linewidth]{hornerplot}
\includegraphics*[width=\linewidth]{decasteljauplot}

\section{Ocena wyników i wnioski}
Z wykresów można wywnioskować, że błąd rośnie wykładniczo. Już nawet przy wielomianie 20-ego stopnia po milionie przekształceń otrzymany błąd jest rzędu $17\%$. Jest to dość duże odchylenie od poprawnej wartości. Stąd ta metoda nie nadaje się do zastosowań, gdzie potrzebna jest duża dokładność obliczeń. Często w praktycznych zastosowaniach w obliczeniach potrzebne są wielomiany stopni znacznie większych niż 20, a będą one jeszcze mniej dokładnie reprezentowane przez błędy arytmetyki komputera. Złożone obrazy wykorzystujące krzywe Beziera korzystają często z wielomianów stopnia około $400$, a nawet większych. W tym wypadku próba przekształcenia postaci Beziera wielomianu do postaci potęgowej skazuje nas na ogromne błędy numeryczne.
\end{document}
