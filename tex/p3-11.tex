\documentclass[wide, 11pt]{mwart}

\usepackage[utf8]{inputenc}
\usepackage[OT4,plmath]{polski}
\usepackage{amsmath,amssymb,amsfonts,amsthm,mathtools,bbm}
 

\date{Wrocław, \today}
\title{Pracownia z analizy numerycznej (M),\\[8pt]\large{Zadanie P3.11}}
\author{Marcin Grzywaczewski\footnote{killavus@gmail.com} \and Szymon Koper\footnote{sz.koper@gmail.com}}

\begin{document}
	\maketitle
	\begin{center}
	Numery indeksów:\\247948 (MG)\\247985 (SzK)
	\end{center}
	\thispagestyle{empty}
	\newpage

\section{Wstęp}
Wielomiany są niezwykle ważną z numerycznego punktu widzenia rodziną funkcji.
Ich unikalne własności pozwalają na wykorzystanie ich w wielu dziedzinach -
od aproksymacji i interpolacji funkcji, po obliczanie wartości całek za pomocą
metod numerycznych. Posiadają one wiele postaci, każda z nich charakteryzuje
się innymi wadami, oraz zaletami.

W tej pracowni zostaną omówione dwie postacie wielomianów - najbardziej 
popularna potęgowa, oraz postać Béziera. Skupiono się na sprawdzenie pod kątem
numerycznym sposobów wzajemnych przekształceń tych postaci z jednej w drugą,
oraz problem obliczania wartości wielomianu w punkcie $x$ w obu tych 
postaciach.

\subsection{Motywacja}
Znana jest bardzo prosta metoda obliczania wartości wielomianu w postaci
potęgowej. Efektywny czasowo algorytm schematu Hornera pozwala na obliczanie
wartości w punkcie $x$ wielomianu za pomocą $n$ mnożeń i $n$ dodawań.

Algorytm opiera się na prostej obserwacji. Wielomian $\pi$ $n$-tego stopnia
\[\pi_n(x) = a_nx^n + a_{n-1}x^{n-1} + \ldots + a_1x + a_0\] możemy zapisać 
jako:
\begin{equation}
  \label{horner_scheme_formula}
  \pi_n(x) = a_0 + x(a_1 + x(a_2 + x(\ldots + x(a_{n-1} + a_nx))\ldots)
\end{equation}

Oczywiście, dla wielomianu $0$ stopnia jego wartością dla dowolnego punktu $x$
jest $a_0$. W przypadku wielomianów stopnia wyższego, mając dane współczynniki
$a_n, a_{n-1}, \ldots, a_1, a_0$, oraz punkt $x$, opierając się na wzorze 
(\ref{horner_scheme_formula}) możemy skonstruować następujący związek 
rekurencyjny:
\begin{subequations}
  \begin{align}
    \beta_0 = \pi_n(x)\\
    \beta_0 = \beta_1 + a_0\\
    \beta_n = a_{n-1} + a_nx\\
    \beta_i = x(a_{n-2} + \beta_{i+1})
  \end{align}
\end{subequations}

Obliczając wartości $\beta_n, \beta_{n-1}, \ldots, \beta_1, \beta_0$ otrzymamy
algorytm obliczania wartości wielomianu w postaci potęgowej w czasie liniowym,
znany pod nazwą \emph{schematu Hornera}.

Niestety, obliczanie wartości wielomianu w postaci potęgowej jest 
słabo uwarunkowane. Sprawia to, że mimo tego, iż posiadamy tak dobry algorytm,
jakim jest schemat Hornera, błąd względny obliczania wartości $\pi_n(x)$ będzie 
duży.

\emph{TODO: Dowód - wskaźnik uwarunkowania obliczania wielomianu w 
postaci potęgowej}

\end{document}
